\documentclass[12pt]{extreport}
\usepackage[T2A]{fontenc}
\usepackage[utf8]{inputenc}        % Кодировка входного документа;
                                    % при необходимости, вместо cp1251
                                    % можно указать cp866 (Alt-кодировка
                                    % DOS) или koi8-r.

\usepackage[english,russian]{babel} % Включение русификации, русских и
                                    % английских стилей и переносов
%%\usepackage{a4}
%%\usepackage{moreverb}
\usepackage{amsmath,amsfonts,amsthm,amssymb,amsbsy,amstext,amscd,amsxtra,multicol}
\usepackage{indentfirst}
\usepackage{verbatim}
\usepackage{tikz} %Рисование автоматов
\usetikzlibrary{automata,positioning}
\usepackage{multicol} %Несколько колонок
\usepackage{graphicx}
\usepackage[colorlinks,urlcolor=blue]{hyperref}
\usepackage[stable]{footmisc}

%% \voffset-5mm
%% \def\baselinestretch{1.44}
\renewcommand{\theequation}{\arabic{equation}}
\def\hm#1{#1\nobreak\discretionary{}{\hbox{$#1$}}{}}
\newtheorem{Lemma}{Лемма}
\newtheorem{Remark}{Замечание}
%%\newtheorem{Def}{Определение}
\newtheorem{Claim}{Утверждение}
\newtheorem{Cor}{Следствие}
\newtheorem{Theorem}{Теорема}
\theoremstyle{definition}
\newtheorem{Example}{Пример}
\newtheorem*{known}{Теорема}
\def\proofname{Доказательство}
\theoremstyle{definition}
\newtheorem{Def}{Определение}

%% \newenvironment{Example} % имя окружения
%% {\par\noindent{\bf Пример.}} % команды для \begin
%% {\hfill$\scriptstyle\qed$} % команды для \end






%\date{22 июня 2011 г.}
\let\leq\leqslant
\let\geq\geqslant
\def\MT{\mathrm{MT}}
%Обозначения ``ажуром''
\def\BB{\mathbb B}
\def\CC{\mathbb C}
\def\RR{\mathbb R}
\def\SS{\mathbb S}
\def\ZZ{\mathbb Z}
\def\NN{\mathbb N}
\def\FF{\mathbb F}
%греческие буквы
\let\epsilon\varepsilon
\let\es\varnothing
\let\eps\varepsilon
\let\al\alpha
\let\sg\sigma
\let\ga\gamma
\let\ph\varphi
\let\om\omega
\let\ld\lambda
\let\Ld\Lambda
\let\vk\varkappa
\let\Om\Omega
\def\abstractname{}

\def\R{{\cal R}}
\def\A{{\cal A}}
\def\B{{\cal B}}
\def\C{{\cal C}}
\def\D{{\cal D}}

%классы сложности
\def\REG{{\mathsf{REG}}}
\def\CFL{{\mathsf{CFL}}}


%%%%%%%%%%%%%%%%%%%%%%%%%%%%%%% Problems macros  %%%%%%%%%%%%%%%%%%%%%%%%%%%%%%%


%%%%%%%%%%%%%%%%%%%%%%%% Enumerations %%%%%%%%%%%%%%%%%%%%%%%%

\newcommand{\Rnum}[1]{\expandafter{\romannumeral #1\relax}}
\newcommand{\RNum}[1]{\uppercase\expandafter{\romannumeral #1\relax}}

%%%%%%%%%%%%%%%%%%%%% EOF Enumerations %%%%%%%%%%%%%%%%%%%%%

\usepackage{xparse}
\usepackage{ifthen}
\usepackage{bm} %%% bf in math mode
\usepackage{color}
%\usepackage[usenames,dvipsnames]{xcolor}

\definecolor{Gray555}{HTML}{555555}
\definecolor{Gray444}{HTML}{444444}
\definecolor{Gray333}{HTML}{333333}


\newcounter{problem}
\newcounter{uproblem}
\newcounter{subproblem}
\newcounter{prvar}

\def\beforPRskip{
	\bigskip
	%\vspace*{2ex}
}

\def\PRSUBskip{
	\medskip
}


\def\pr{\beforPRskip\noindent\stepcounter{problem}{\bf \theproblem .\;}\setcounter{subproblem}{0}}
\def\pru{\beforPRskip\noindent\stepcounter{problem}{\bf $\mathbf{\theproblem}^\circ$\!\!.\;}\setcounter{subproblem}{0}}
\def\prstar{\beforPRskip\noindent\stepcounter{problem}{\bf $\mathbf{\theproblem}^*$\negthickspace.}\setcounter{subproblem}{0}\;}
\def\prpfrom[#1]{\beforPRskip\noindent\stepcounter{problem}{\bf Задача \theproblem~(№#1 из задания).  }\setcounter{subproblem}{0} }
\def\prp{\beforPRskip\noindent\stepcounter{problem}{\bf Задача \theproblem .  }\setcounter{subproblem}{0} }

\def\prpvar{\beforPRskip\noindent\stepcounter{problem}\setcounter{prvar}{1}{\bf Задача \theproblem \;$\langle${\rm\Rnum{\theprvar}}$\rangle$.}\setcounter{subproblem}{0}\;}
\def\prpv{\beforPRskip\noindent\stepcounter{prvar}{\bf Задача \theproblem \,$\bm\langle$\bracketspace{{\rm\Rnum{\theprvar}}}$\bm\rangle$.  }\setcounter{subproblem}{0} }
\def\prv{\beforPRskip\noindent\stepcounter{prvar}{\bf \theproblem\,$\bm\langle$\bracketspace{{\rm\Rnum{\theprvar}}}$\bm\rangle$}.\setcounter{subproblem}{0} }

\def\prpstar{\beforPRskip\noindent\stepcounter{problem}{\bf Задача $\bf\theproblem^*$\negthickspace.  }\setcounter{subproblem}{0} }
\def\prdag{\beforPRskip\noindent\stepcounter{problem}{\bf Задача $\theproblem^{^\dagger}$\negthickspace\,.  }\setcounter{subproblem}{0} }
\def\upr{\beforPRskip\noindent\stepcounter{uproblem}{\bf Упражнение \theuproblem .  }\setcounter{subproblem}{0} }
%\def\prp{\vspace{5pt}\stepcounter{problem}{\bf Задача \theproblem .  } }
%\def\prs{\vspace{5pt}\stepcounter{problem}{\bf \theproblem .*   }
\def\prsub{\PRSUBskip\noindent\stepcounter{subproblem}{\sf \thesubproblem .} }
\def\prsubr{\PRSUBskip\noindent\stepcounter{subproblem}{\bf \asbuk{subproblem})}\;}
\def\prsubstar{\PRSUBskip\noindent\stepcounter{subproblem}{\rm $\thesubproblem^*$\negthickspace.  } }
\def\prsubrstar{\PRSUBskip\noindent\stepcounter{subproblem}{$\text{\bf \asbuk{subproblem}}^*\mathbf{)}$}\;}

\newcommand{\bracketspace}[1]{\phantom{(}\!\!{#1}\!\!\phantom{)}}

\DeclareDocumentCommand{\Prpvar}{ O{null} O{} }{
	\beforPRskip\noindent\stepcounter{problem}\setcounter{prvar}{1}{\bf Задача \theproblem
% 	\ifthenelse{\equal{#1}{null}}{  }{ {\sf $\bm\langle$\bracketspace{#1}$\bm\rangle$}}
%	~\!\!(\bracketspace{{\rm\Rnum{\theprvar}}}).  }\setcounter{subproblem}{0}
%	\;(\bracketspace{{\rm\Rnum{\theprvar}}})}\setcounter{subproblem}{0}
%
	\,{\sf $\bm\langle$\bracketspace{{\rm\Rnum{\theprvar}}}$\bm\rangle$}
	~\!\!\! \ifthenelse{\equal{#1}{null}}{\!}{{\sf(\bracketspace{#1})}}}.

}
%\DeclareDocumentCommand{\Prpvar}{ O{level} O{meta} m }{\prpvar}


\DeclareDocumentCommand{\Prp}{ O{null} O{null} }{\setcounter{subproblem}{0}
	\beforPRskip\noindent\stepcounter{problem}\setcounter{prvar}{0}{\bf Задача \theproblem
	~\!\!\! \ifthenelse{\equal{#1}{null}}{\!}{{\sf(\bracketspace{#1})}}
	 \ifthenelse{\equal{#2}{null}}{\!\!}{{\sf [\color{Gray444}\,\bracketspace{{\fontfamily{afd}\selectfont#2}}\,]}}}.}

\DeclareDocumentCommand{\Pr}{ O{null} O{null} }{\setcounter{subproblem}{0}
	\beforPRskip\noindent\stepcounter{problem}\setcounter{prvar}{0}{\bf\theproblem
	~\!\!\! \ifthenelse{\equal{#1}{null}}{\!\!}{{\sf(\bracketspace{#1})}}
	 \ifthenelse{\equal{#2}{null}}{\!\!}{{\sf [\color{Gray444}\,\bracketspace{{\fontfamily{afd}\selectfont#2}}\,]}}}.}

%\DeclareDocumentCommand{\Prp}{ O{level} O{meta} }

\DeclareDocumentCommand{\Prps}{ O{null} O{null} }{\setcounter{subproblem}{0}
	\beforPRskip\noindent\stepcounter{problem}\setcounter{prvar}{0}{\bf Задача $\bm\theproblem^* $
	~\!\!\! \ifthenelse{\equal{#1}{null}}{\!}{{\sf(\bracketspace{#1})}}
	 \ifthenelse{\equal{#2}{null}}{\!\!}{{\sf [\color{Gray444}\,\bracketspace{{\fontfamily{afd}\selectfont#2}}\,]}}}.
}

\DeclareDocumentCommand{\Prpd}{ O{null} O{null} }{\setcounter{subproblem}{0}
	\beforPRskip\noindent\stepcounter{problem}\setcounter{prvar}{0}{\bf Задача $\bm\theproblem^\dagger$
	~\!\!\! \ifthenelse{\equal{#1}{null}}{\!}{{\sf(\bracketspace{#1})}}
	 \ifthenelse{\equal{#2}{null}}{\!\!}{{\sf [\color{Gray444}\,\bracketspace{{\fontfamily{afd}\selectfont#2}}\,]}}}.
}


\def\prend{
	\bigskip
%	\bigskip
}




%%%%%%%%%%%%%%%%%%%%%%%%%%%%%%% EOF Problems macros  %%%%%%%%%%%%%%%%%%%%%%%%%%%%%%%



%\usepackage{erewhon}
%\usepackage{heuristica}
%\usepackage{gentium}

\usepackage[portrait, top=3cm, bottom=1.5cm, left=3cm, right=2cm]{geometry}

\usepackage{fancyhdr}
\pagestyle{fancy}
\renewcommand{\headrulewidth}{0pt}
\lhead{\fontfamily{fca}\selectfont {Алгоритмы. Пономаренко Алексей.} }
%\lhead{ \bf  {ТРЯП. } Семинар 1 }
%\chead{\fontfamily{fca}\selectfont {Вариант 1}}
\rhead{\fontfamily{fca}\selectfont Домашнее задание 6}
%\rhead{\small 01.09.2016}
\cfoot{}

\usepackage{titlesec}
\titleformat{\section}[block]{\Large\bfseries\filcenter {\setcounter{problem}{0}}  }{}{1em}{}


%%%%%%%%%%%%%%%%%%%%%%%%%%%%%%%%%%%%%%%%%%%%%%%%%%%% Обозначения и операции %%%%%%%%%%%%%%%%%%%%%%%%%%%%%%%%%%%%%%%%%%%%%%%%%%%% 
                                                                    
\newcommand{\divisible}{\mathop{\raisebox{-2pt}{\vdots}}}           
\let\Om\Omega


%%%%%%%%%%%%%%%%%%%%%%%%%%%%%%%%%%%%%%%% Shen Macroses %%%%%%%%%%%%%%%%%%%%%%%%%%%%%%%%%%%%%%%%
\newcommand{\w}[1]{{\hbox{\texttt{#1}}}}

\newcommand{\comments}[2][Комментарий]{
\medskip
	\noindent{\bfseries #1: }{\textsl{#2}}
%\medskip	
}

\begin{document}

\Pr \hspace{1mm} Чтобы реализовать стек, используя две очереди, проделаем следующие шаги. Для реализации нам необходимо "выразить" все команды стека через команды очереди. У стека есть 2 команды: PUSH() и POP(). Действуем так: если пользователь хочешь запушить элемент, мы смотрим на первую очередь: если она пустая - просто кладем в нее элемент, а если нет - все содержимое перекладываем во вторую очередь, кладем наш элемент в первую, а затем кладем из второй в первую все элементы из второй. Таким способом мы формируем из входной последовательности полностью реверснутую, а значит для операции POP() нам нужно будет просто взять первый элемент перво очереди. Итак, мы реализовали команды PUSH() и POP(), теперь выясним, с какой сложностью это проделано. Очевидно, POP() - за $O(1)$, потому что мы просто достаем первый элемент очереди. А вот PUSH() - уже за $O(n)$, так как в худшем случае нам придется 2 раза перегонять $(n - 1)$ элемент сначала из 1 во 2 очередь, а затем обратно.

\Pr \hspace{1mm} Хранить почти-полное троичное дерево в массиве можно следующим образом. Нумеруем вершины дерева сверху-вниз слева-направо \underline{от 1}, то есть корень дерева это 1, его левый сын - 2, центральный - 3, правый - 4, и так далее. В таком случае номер левого ребенка это (номер родителя * 3 - 1), номер центрального это (номер родителя * 3), а номер правого - это (номер родителя * 3 + 1). В обратную сторону так: если (номер ребенка + 1) делится на 3, то номер родителя это частное от этого деления, если делится нацело, то это число и есть номер родителя, иначе если (номер ребенка - 1) делится на 3, то номер родителя это частное от этого деления.

\Pr \hspace{1mm} Сначала формируем по исходному массиву кучу по убыванию за $O(n)$. Затем последовательно $k$ раз извлекам минимальный элемент из кучи. Итого, получаем сложноть $O(n + k * \log_2(n)$

\Pr \hspace{1mm} Нам известно, что $y > x$, причем между ними нет ни одного числа, принадлежащего дереву, а также правое поддерево $x$ пустое. Это значит, что $x$ нах-ся где-то в левом поддереве эл-та $y$. Таким образом, мы пока что доказали, что $y$ это предок $x$. Далее возможны 2 варианта: либо само число $x$ нах-ся непосредственно сразу за $y$ без промежуточных эл-тов, тогда $y$ - самый нижний предок, чей левый дочерний узел явл-ся самим $x$; либо в левом узле от $y$ находится число, которое меньше, чем $x$, а тогда путь к $x$ будет проходить через $y$ и этот элемент. Таких элементов может быть несколько, но все они будут строго меньше $x$. Например, $5-1-2-3-4$, где $y = 5, x = 4$. Тогда при последовательном посроении дерева между $y$ и $x$ будут лежать аж 3 эл-та (аналогично, между ними на пути может лежать произвольное число элементов, удовлетворяющих данному неравенству: $elem < x < y$), но при этом $y$ будет самым нижним предком, чей левый дочерний узел явл-ся предком $x$. 

\Pr \hspace{1mm} Рассмотрим левый узел. Пойдем от противного. Пусть у левой дочерней вершины \underline{есть правое поддерево}. Тогда, все элементы этого поддерева удовлетворяют неравенству $a.key < elem < b.key$, так как лежат в правом поддереве $a$ (а значит больше самого $a$), но при этом в левом поддереве $b$ (а значит меньше $b$). Это противоречит условию задачи о том, что между $a.key$ и $b.key$ не лежит ни одно число. Это значит, что у левой дочерней вершины нет правого поддерева. Абсолютно аналогично с правым поддеревом вершины $b$.

\Pr \hspace{1mm} Двоичное дерево поиска - дерево, в левом поддереве которого лежат элементы, меньшие его, а в правом - большие. Построить такое дерево по произвольной входной последовательности $=$ отсортировать массив входных данных за линейное время, что невозможно.

\Pr \hspace{1mm} Время ожидания каждого клиента будет минимальным, если обслуживать клиентов в порядке возрастания времени на их обслуживание, потому что время ожидания каждого клиента это сумма всего времени на обслуживание всех предыдущих, а эту сумму мы можем минимизировать за счет минимизации каждого слагаемого, причем начиная уже с самого короткого по обслуживанию человека. То есть, чтобы второй по времени ожидания человек ждал минимально, первым надо обслужить самого быстрого. Также это можно объяснить так: для самого последнего обслуживаемого клиента не важно, в каком порядке обслуживали всех предыдущих, потому что его личное время ожидания при этом не изменится. Но уже для минимизации времени ожидания предпоследнего клиента очень важно, чтобы клиент с самым долгим временем обслуживания был после него, и так далее. Теперь нужно предложить сам алгоритм, который наиболее эффективно отсортирует по возрастанию время обслуживания каждого клиента, и это и будет ответом на то, в каком порядке нужно обслуживать клиентов. Для этого воспользуемся сортировкой "HeapSort", сложность которой равна $O(n * \log_2(n)$. 

\end{document}